%%%%%%%%%%%%%%%%%%%%%%%%%%%%%%%%%%%%%%%%%%%%%%%%%%%%%%%%%%%%%%%%%%%%%%%%%%%%%%%%%%%
% 			Facultad de Ciencias, UAEM.							Agosto de 2013
% 
%	Alumno: 				Emanuel García Perez
%	Asginatura:				Computación y Sociedad
%	Proyecto:				Exposición
%	Tema:					"Censura en Internet // OpenNet Initiative"
%
%%%%%%%%%%%%%%%%%%%%%%%%%%%%%%%%%%%%%%%%%%%%%%%%%%%%%%%%%%%%%%%%%%%%%%%%%%%%%%%%%%%


\documentclass{beamer}

\usepackage[spanish,activeacute]{babel}
\usepackage[latin1]{inputenc}
\usepackage{beamerthemeshadow}
\usepackage{graphicx}

\title{\textbf{Censura en Internet // OpenNet Initiative}}
\author{Emanuel Garc\'ia P\'erez}
\date{\today}

\begin{document}

\frame[allowframebreaks]{\titlepage}

\section[Contenidos]{}
\frame{\tableofcontents}


\section{Censura}

\frame
{
\frametitle{?`Qu\'e es la Censura?}
Practica que implica la supresi\'on de contenido de un material de comunicaci\'on por razones ideol\'ogicas, morales o pol\'iticas; cuando cierto contenido es considerado ofensivo, da\~nino, inconveniente o innecesario suele ser censurado por un medio de comunicaci\'on, el gobierno o cualquier otro organismo que as\'i lo determine.
}

\subsection{Justificaciones}

\frame{
\frametitle{Contextos de Censura}

\transdissolve[duration=0.2]
	\begin{itemize}
		\item<1-> Item 1
	\end{itemize}
}



\end{document}
